\chapter{短波语音客观质量评价算法}
\label{chap:algorithms}

为了能够从多路短波语音中自动切换选择,需要使用算法对各路短波语音的质量进行评价。由于系统没有原始语音作为参考,所以需要的客观质量评价算法是单端的客观评价算法。现有的语音客观质量评价算法主要针对VoIP应用、语音编解码系统、语音降噪系统等场景,这些场景中的语音质量相较短波语音要好,信噪比较高。直接将现有的算法应用到短波语音上难以取得良好的评价结果。所以本章提出了两种可用在短波语音上的客观质量评价算法

\section{基于语谱图噪音模型的客观质量评价算法}

\subsection{启发性思路} \label{section:alg1-1}

语谱图是一种从频域对语音信号进行分析的工具,将语音信号转换到频域进行分析,需要使用短时傅里叶变换(STFT)。给定一个时间宽度很短的窗函数,语音信号的STFT定义为\ref{eq:stft}

\begin{equation}\label{eq:stft}
F_{STFT}(t, f) = \int_{-\infty}^{+\infty}x(u)g^*(u-t)e^{-j2\pi fu}du
\end{equation}

由于窗函数在时域有限,使得短时傅里叶变换具有局部特性,能够反映指定时刻的局部频谱信息。
实际应用中,需要使用离散化的短时傅里叶变换,取$t=mT, f=nF$,其中$T$和$F$分别为时间和频率的采样间隔,$m,n=0,1,2,...,M-1$,$M$是采样点数。则离散形式的STFT为\ref{eq:dstft}

\begin{equation}\label{eq:dstft}
F_{STFT}(m, n) = \sum_{k=0}^{N-1}x(kT)g^*((k-m)T)e^{-j2\pi nk/N}
\end{equation}

$F_{STFT}(m, n)$反映的是$mT$时刻,$nF$频率附近的局部频谱信息。

根据对于人耳听觉特性的一些研究,人耳对于声音信号的相位特性并不敏感,但是对于幅度特性敏感,并且人耳对于声音响度的感受是与信号能量的对数成正比的。所以对离散短时傅里叶变换得到的结果,只保留幅值能量,并进一步做取对数的处理,得到的二维信息可以用图像展示,称之为语谱图,定义为\ref{eq:log_dstft}

\begin{equation}\label{eq:log_dstft}
G(m, n) = log(|F_{STFT}(m, n)|^2)
\end{equation}

语谱图反映了语音信号在不同时间、不同频率上的能量大小,承载了语音所包含的信息。高质量语音的语谱图上有着清晰的结构特征,包括基音部分的能带,各种高次谐波的能带;而带有噪音的语谱图则没有模糊不清。在短波语音信号上亦是如此。如图~\ref{fig:timefreq}所示,从左到右是质量由低到高五组短波语音的语谱图,分别对应主观评分1-5分。从语谱图上可以看到条带状的语音部分越来越突出和明显。图~\ref{fig:timefreq-thr0.3}是由图~\ref{fig:timefreq}中各个语谱图以0.3为阈值转化为二值图像的结果,图中可以更加明显的看出质量越好的语音信号,条带状的纯净语音部分越清楚,而质量较差的语音信号,则因为存在噪音连成模糊的一片。

\begin{figure}
\centering
\includegraphics[width=1\textwidth]{timefreq}
\caption{不同质量的短波语音语谱图\label{fig:timefreq}}
\end{figure}

\begin{figure}
\centering
\includegraphics[width=1\textwidth]{timefreq-thr}
\caption{不同质量的短波语音二值化语谱图\label{fig:timefreq-thr0.3}}
\end{figure}

\begin{figure}
\centering
\includegraphics[width=1\textwidth]{timefreq-thrs-label}
\caption{不同阈值的二值化语谱图\label{fig:timefreq-thrs-label}}
\end{figure}

进一步地,从中选出一个语音信号,分别以0.2, 0.4, 0.6, 0.8为阈值绘制其二值化语谱图,如图~\ref{fig:timefreq-thrs-label}所示。而图中绿色区域是程序通过建立噪音模型,使用一些特征标注出的被噪音干扰的区域。可以发现,在阈值较低时,因为噪音部分留在了图上,所以看上去是大片大片的噪音;而当阈值逐渐升高,噪音部分逐渐从图上消失,留下的就都是纯净的语音信号。若当阈值超过某一阈值t时,语谱图上某一点不再被噪音干扰,那么这一点实际能量比t高出的部分就表征了这一点语音信号相对于噪音部分高出的强度。本文称之为该点的语音分辨率,计算语谱图上所有点的语音分辨率,进而得出语音质量的客观评价分。

\subsection{算法流程}

本小节介绍基于语谱图噪音模型的客观质量评价算法的具体流程,算法总体流程示意如图~\ref{fig:flowchart}所示。一维的短波语音信号经过加窗傅里叶变换得到二维的频域信号,其中横轴为时间维度,纵轴为频率维度。随后用不同阈值将语谱图转化成多张二值图像,再各个二值图像上使用图像处理的一些方法区分噪音区域和语音区域。再此基础上得到原始语谱图上各个点的“语音分辨率”,进而得到对于短波语音的客观评价分数。下面详细介绍各个步骤的内容。

\begin{figure}
\centering
\includegraphics[width=0.8\textwidth]{flowchart}
\caption{基于语谱图噪音模型的客观质量评价算法流程\label{fig:flowchart}}
\end{figure}

首先根据上一小节介绍的语谱图的内容,使用短时傅里叶变换获得短波语音信号的语谱图$G(m,n)$。

取一系列的阈值$t_1<t_2<t_3<...<t_N$,我们进一步对语谱图做阈值化处理形成二值的语谱图,阈值化语谱图定义为

\begin{equation}\label{eq:thr_tf}
G_i(m, n) = \left\{
    \begin{array}{rcl}
    1, && {G(m,n)>t_i} \\
    0, && {G(m,n)\leq t_i}
    \end{array} \right.
\end{equation}

根据短波语音语谱图上的噪音模型,使用以下几种条件来判定噪音区域:
\begin{enumerate}
\item 区域$A_1$:语谱图上通过霍夫变换检测的与水平夹角小于1度的直线。
\item 区域$A_2$:某一长宽各大于一定阈值的范围内,$G_i(m,n)=1$的点数占据面积的80\%以上。
\item 区域$A_3$:某一连通域面积比上该连通域占据的宽度,其值超过一定阈值。
\item 区域$A_4$:某一连通域,其面积小于一定阈值。
\end{enumerate}

标注后的语谱图
\begin{equation}\label{eq:label_tf}
G_i^*(m, n) = \left\{
    \begin{array}{rcl}
    0, && {G_i(m,n)=0} \\
    1, && {G_i(m,n)=1且(m,n)\in \bigcup_{k=1}^4 A_k} \\
    2, && {else}
    \end{array} \right.
\end{equation}

定义集合
\begin{equation}\label{eq:collection}
\Psi(m,n)={i|G_i*(m,n)=2}
\end{equation}

定义语音分辨率
\begin{equation}\label{eq:resolution}
D(m,n) =  \left\{
    \begin{array}{rcl}
    0, && {\Psi(m,n)=\emptyset} \\
    G(m,n) - \min_{i\in\Psi(m,n)}t_i, && {\Psi(m,n)\neq\emptyset}
    \end{array} \right.
\end{equation}

对语音分辨率取平均得到客观评价分数
\begin{equation}\label{eq:score}
S_1 = \frac{\sum D(m,n)}{\sum G_1(m,n)}
\end{equation}

为保证阈值化语谱图包含有效信息,阈值参数的取值范围应该是。一种简单的选取方法是直接在到选取$N$等分点。
为了尽量细致地刻画语谱图的特征,阈值的选取应该尽量密集,但同时算法的时间复杂度正比于阈值的数量,密集的阈值也会导致更高的计算复杂度。所以阈值的选取应当使用尽量少的阈值尽量高效地刻画语谱图的特征,上述选取方法显然不够高效。一方面,阈值很接近$\min{G(m,n)}$或者$\max{G(m,n)}$时,阈值化语谱图上的点非常少或者非常多,能提供的信息量很少。另一方面因为阈值增加同样大小,阈值化语谱图上的点数增加可能差异很大。均匀地选取阈值,会导致在有些区间,连续的几张阈值化语谱图过于相像而浪费计算资源,有些区间,阈值化语谱图又变化太快导致刻画不够细致。

一种改进的动态选取阈值的方法如下:先在5\%至20\%之间均匀选定一系列比例$\alpha_1,\alpha_2,\alpha_3,...,\alpha_N$,再计算得出对应的阈值$t_1,t_2,t_3,...,t_N$使得$\sum G_i(m,n)=\alpha_i M^2$。实验验证此种阈值选取方法要比前述方法效果提升很多。


\section{基于自编码器的客观质量评价算法}

上节介绍的基于语谱图噪音模型的客观质量评价算法虽然能够很好的评价短波语音的质量,实验也表明算法给出的评分能够反映人的主观感受。但是该算法是基于人为经验分析的语谱图噪音模型,在应用到其他领域的低信噪比语音时,由于可能存在的噪音不一样,需要重新分析噪音的模型特征。所以该算法针对性很强,而适用性则较窄,迁移应用较难。

为此,本节从另一个角度出发,通过人工神经网络的自编码器学习纯净短波语音的语谱特征,对语音部分而非噪音部分建模,然后再使用自编码器来评价短波语音的质量。由于人工神经网络的学习可以自动完成,不需要过多人工干预,所以这种方法可以方便的迁移到其他领域的应用中。

\subsection{梅尔频谱语谱图}

~\ref{section:alg1-1}小节中介绍的语谱图在应用到自编码器中时,频率方向采样太密,使得输入向量长度太长,所以我们希望对频率方向降采样。由于人耳的听觉特性在频率方向并非线性,人耳在低频区域有更多的感受器,对频率的分辨相对灵敏,而随着频率升高,对频率的分辨逐渐降低。所以降采样时我们使用根据这种人耳听觉特性而设计的非线性的滤波器组——梅尔滤波器组。

梅尔滤波器组来源于语音信号处理中常用的梅尔频率倒谱系数(MFCC)的计算过程。滤波器数量我们设定为40,使用40个三角滤波器$H_k, k=0,1,2...,39$组成的滤波器组来处理原始语谱图。每个滤波器的输出对应于一个频带的能量,40个频带的中心频率分布是非线性的,在低频处分布较密集,高频处分布较疏松。每个滤波器中心频率和$k$的关系为~\ref{eq:mel-freq},各个滤波器的冲击响应$H_k$满足~\ref{eq:mel-filters}。

\begin{equation}\label{eq:mel-freq}
f(k) = 700(10^{k/2595}-1)
\end{equation}

\begin{equation}\label{eq:mel-filters}
H_k(n) = \left\{
    \begin{array}{rcl}
    0, && {n<f(k-1)} \\
    \frac{n-f(k-1)}{f(k)-f(k-1)}, && {f(k-1)\leq n < f(k)} \\
    1, && {n=f(k)} \\
    \frac{f(k+1)-n}{f(k+1)-f(k)}, && {f(k) < n \leq f(k+1)} \\
    0, && {n > f(k+1)}
    \end{array} \right.
\end{equation}

使用这40个三角滤波器组成的梅尔滤波器组,我们得到更符合人耳听觉特性的梅尔频谱(Mel-Frequency)语谱图M(m, k),其中$m=0,1,2,...M-1; k=0,1,2,…39$。

\subsection{自编码器结构}

由于语音存在一定的结构特征,所以语音频谱中存在冗余信息,可以通过更少的信息来表示。人工神经网络构建的自编码器(Autoencoder),可以用来提取高维向量中的结构化特征,将其压缩到较低维空间表示。
对8kHz的短波语音使用32ms的窗长加汉明窗,以16ms的长度移动汉明窗,使用40个梅尔频谱滤波器,生成梅尔频谱语谱图。取连续7帧语音的语谱图,拼接成自编码器的输入向量,向量长度为7*40 = 280。自编码器的隐藏层大小设置为80,网络结构如图~\ref{fig:autoencoder}所示。

\begin{figure}
\centering
\includegraphics[width=0.8\textwidth]{autoencoder}
\caption{自编码器网络结构\label{fig:autoencoder}}
\end{figure}

自编码器网络使用tensorflow\cite{abadi2016tensorflow}实现,隐藏层激活函数使用softplus函数$\sigma(x)=ln⁡(e^x+1)$,使用质量非常高、基本未受到干扰或者噪音污染的短波语音训练自编码器,使其学习短波语音的语谱结构特征。使用总计时长约5小时的清晰短波语音进行训练,每个训练样本的语音时长为16*8=128ms,故总计样本数量为14万左右。训练的batch大小设置为256,初始学习速率0.001。


\begin{figure}
\begin{minipage}{0.55\textwidth}
  \centering
  \includegraphics[width=1\textwidth]{ae-pure1}
  \caption{自编码器输入输出语谱图(清晰语音)\label{fig:ae-pure1}}
\end{minipage}\hfill
\begin{minipage}{0.35\textwidth}
  \centering
  \includegraphics[width=1\textwidth]{ae-sys}
  \caption{自编码器评分流程 \label{fig:ae-sys}}
\end{minipage}
\end{figure}

经过一段时间的训练,自编码器可以学习到清晰短波语音的语谱特征,如图~\ref{fig:ae-pure1}是一段清晰短波语音经过自编码器的输入输出结果。可以看出,尽管自编码器将输入向量的长度由输入的280压缩到了隐藏层的80,最终输入和输出的梅尔语谱图相差甚微,基本的特征都得到了保留。

\subsection{自编码器评分}

经过训练的自编码器记住了清晰短波语音的语谱结构特征,所以当输入的短波语音是质量很高,很清晰的语音时,输出的语谱图和输入的语谱图相差甚微。但是如果输入的短波语音是在传播中发生畸变的,或者受到噪音污染的,因为自编码器模型并没有学习畸变或者噪音的语谱结构,输出的语谱图会和输入的语谱图有所差别。

使用自编码器对短波语音进行客观质量评分的过程如图~\ref{fig:ae-sys}所示,我们通过分析经过自编码器所带来的语谱图变化的大小来给短波语音质量评分,变化越大说明输入语音的质量越差。图~\ref{fig:ae-low} \~ ~\ref{fig:ae-high}分别为质量较低、中等、较高的语音语谱图,左侧为原始语谱图,右侧为经过自编码器之后的语谱图,从中可以明显看出语音质量越高,语谱图经过自编码器之后的变化越小。

\begin{figure}
\centering
\subcaptionbox{低质量语音\label{fig:ae-low}} {
    \includegraphics[width=0.45\textwidth]{ae-low}
}
\subcaptionbox{中等质量语音\label{fig:ae-mid}}{
    \includegraphics[width=0.45\textwidth]{ae-mid}
}
\vspace{0.8ex}
\\
\subcaptionbox{高质量语音\label{fig:ae-high}} {
    \includegraphics[width=0.45\textwidth]{ae-high}
}
\caption{不同质量语音经过自编码器时语谱图的变化\label{fig:ae-diffs}}
\end{figure}

由于经过自编码器后的语谱图整体幅值均值可能有变化,所以直接对输入输出语谱图作差来计算语谱图的差异不可取。语谱图的信息主要体现在各个区域的相对幅值大小上,以及幅值较高部分的形状信息上。所以我们定义如下的语谱图差异函数:

\begin{equation}
D = \max_{q \in Q}\left(\sum_{m, k} (M_{in}(m,k) > P_{in}(q)) \bigoplus (M_{out}(m,k) > P_{out}(q))\right)
\end{equation}

其中$Q={70,72,74,76,78,80,82,84,86}$,$M_{in}$和$M_{out}$分别为输入输出的梅尔语谱图,$P_{in}$和$P_{out}$分别为输入输出语谱图的百分位函数,例如$P_{in}(70)$表示输入语谱图的70th百分位数。

基于此差异函数即可进一步给出短波语音的客观评分$S_2$如公式~\ref{eq:s2}。

\begin{equation}\label{eq:s2}
S_2 = \frac{40M - D}{40M}
\end{equation}

\section{小结}

本章主要介绍了两种适用于短波语音的客观质量评价算法。一种是基于语谱图噪音模型的算法,通过分析短波语音语谱图,建立噪音模型,分析计算语谱图上的“语音分辨率”对短波语音进行客观评分。另一种是基于自编码器的算法,通过几乎未被噪音污染的高质量的短波语音训练自编码器,使得自编码器记录学习纯净短波语音的语谱结构特征,再根据待评价语音通过自编码器时的变化大小来对短波语音进行客观评分。这两种算法在短波语音上都有很好的表现,为接下来的自动选路系统做好了准备。