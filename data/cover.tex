\thusetup{
  %******************************
  % 注意:
  %   1. 配置里面不要出现空行
  %   2. 不需要的配置信息可以删除
  %******************************
  %
  %=====
  % 秘级
  %=====
  secretlevel={绝密},
  secretyear={2100},
  %
  %=========
  % 中文信息
  %=========
  ctitle={短波语音信号的质量评价算法及自动选路系统研究},
  cdegree={工学硕士},
  cdepartment={电子工程系},
  cmajor={信息与通信工程},
  cauthor={陈晔},
  csupervisor={谷源涛副教授},
  %cassosupervisor={陈文光教授}, % 副指导老师
  %ccosupervisor={某某某教授}, % 联合指导老师
  % 日期自动使用当前时间,若需指定按如下方式修改:
  cdate={二〇一七年六月},
  %
  % 博士后专有部分
  %cfirstdiscipline={计算机科学与技术},
  %cseconddiscipline={系统结构},
  %postdoctordate={2009年7月——2011年7月},
  %id={编号}, % 可以留空: id={},
  %udc={UDC}, % 可以留空
  %catalognumber={分类号}, % 可以留空
  %
  %=========
  % 英文信息
  %=========
  etitle={Study on Preprocessing Technique and Background Model in Drosophila Recognition and Behavior Analysis},
  % 这块比较复杂,需要分情况讨论:
  % 1. 学术型硕士
  %    edegree:必须为Master of Arts或Master of Science(注意大小写)
  %             “哲学、文学、历史学、法学、教育学、艺术学门类,公共管理学科
  %              填写Master of Arts,其它填写Master of Science”
  %    emajor:“获得一级学科授权的学科填写一级学科名称,其它填写二级学科名称”
  % 2. 专业型硕士
  %    edegree:“填写专业学位英文名称全称”
  %    emajor:“工程硕士填写工程领域,其它专业学位不填写此项”
  % 3. 学术型博士
  %    edegree:Doctor of Philosophy(注意大小写)
  %    emajor:“获得一级学科授权的学科填写一级学科名称,其它填写二级学科名称”
  % 4. 专业型博士
  %    edegree:“填写专业学位英文名称全称”
  %    emajor:不填写此项
  edegree={Master of Science},
  emajor={Information and Communication Engineering},
  eauthor={Hou Qi},
  esupervisor={Associate Professor Gu Yuantao},
  %eassosupervisor={Chen Wenguang},
  % 日期自动生成,若需指定按如下方式修改:
  edate={April, 2016},
  %
  % 关键词用“英文逗号”分割
  ckeywords={果蝇, 背景模型, 行为分析},
  ekeywords={Drosophila, Background Model, Behavior Analysis}
}

% 定义中英文摘要和关键字
\begin{cabstract}
  动物行为学研究是生物学研究的一个重要领域,动物行为学通过观察、计数研究动物的行为。果蝇的基因结构简单、身体结构简单,成为动物行为学研究中的重点研究对向。果蝇行为学研究需要分析大量的果蝇视频数据,手工标定需要花费大量的人力成本,且手工标定可能存在标准不一致等问题,用计算机自动分析果蝇的行为成为一个可行解决方案。目前已经存在自动分析果蝇行为的研究。果蝇行为识别的主要流程包括果蝇活动台提取、果蝇身体轮廓提、果蝇特征提取、行为模式识别等步骤。目前的研究一般受限于特定的实验条件和拍摄环境,往往难以适应其他实验环境。其中,果蝇轮廓提取步骤往往是其中的制约因素。此外,果蝇活动台提取步骤也限制了整个流程自动化的程度。为此,本文提出一种特定排列模式下的果蝇活动台提取算法和一种基于背景模型的轮廓提取算法。主要工作可以概括为以下几个方面:

  首先,本文提出一种特定排列模式下的果蝇活动台轮廓提取算法。通过圆检测等形状检测方式提取活动台的位置和大小,进而根据果蝇活动台的排列方式,对果蝇活动台的位置进行进一步的调整。

  其次,本文提出一种基于背景模型的果蝇轮廓提取算法,可以从视频中提取果蝇的身体和翅膀。算法采用单高斯模型作为背景模型,通过亮度畸变区分果蝇翅膀和身体。初步建模后分离果蝇身体和活动台背景;然后在对活动台单独进建模的基础上,得到最终的背景模型。

  然后,在背景模型的基础上,进行果蝇打架行为分析,验证了本文提出的算法的有效性。

  最后,搭建果蝇行为识别网站,并介绍果蝇行为识别软件的程序部分和网站部分,以及网站的基本使用说明。

  本文提出的基于特定排列模式的果蝇活动台算法提高了果蝇自动分析的自动化程度;基于背景模型的果蝇轮廓提取算法具有较好的鲁棒性,可以适用于不同拍摄环境,有助于提高自动化果蝇行为分析的适用范围。此外,该模型还可以应用于动物行为学中对其他动物的研究。


\end{cabstract}

% 如果习惯关键字跟在摘要文字后面,可以用直接命令来设置,如下:
% \ckeywords{\TeX, \LaTeX, CJK, 模板, 论文}

\begin{eabstract}

In biology research, ethology is an important area. By observing the behavior of animals, ethologists get the internal mechanism of animals. Drosophila has a simple genetic structure as well as a simple body structure, which makes it an important role in ethology. In traditional research, much manpower is needed to watch the videos and count the behaviors. Also, there may be different standards for behavior detection, which makes it hard to generalize. With the development of computer vision and machine learning, analyzing the fly video by computer becomes possible.

The main procedure includes drosophila chamber extraction, drosophila body contour extraction, drosophila feature extraction, and drosophila behavior pattern recognition. But the current study is generally limited by specific environments, and cannot generalize to other laboratory circumstance. Drosophila contour extraction step is often the restrictive factors, followed by drosophila chamber extraction step. To solve this problem, we propose a method to extract drosophila chambers based on the specific layout pattern. For the contour extraction step, we propose a drosophila contour extraction algorithm based on background model. The main work can be summarized as follows:

Firstly, this paper presents a drosophila chamber extraction algorithm based on specific layout pattern. First extract the chambers by shape detection method such as hough circle detection, then cluster the shapes to get the position of chambers, then adjust the position according to the layout of the chambers with transformation of coordinates.

Secondly, this paper presents a drosophila contour extraction algorithm based on background model, which extracts the body and wings of drosophila from the video. The algorithm uses a single Gaussian distribution as the background model, and distinguishes the drosophila wings and body by luminance distortion. After the preliminary background modeling, we can separate the drosophila body from the containers. Then we do background modeling on the container only, which makes a better background model for drosophila contour extraction.

Then, we analyze the fly fighting behaviors based on the algorithm, and it turns out to be effective.

Finally, we build a website that provide fly behavior analysis service, also we make a detailed introduction to the DetectFly software and the website, as well as the basic instructions for the website.

In conclusion, the drosophila chamber extraction algorithm reduce the manual operation in drosophila analysis. The contour extraction algorithm proves to be robust in different videos from different cameras, which helps to improve the application scope of  automated drosophila behavior analysis. In addition, the model can be applied to animal studies in other animals.

\end{eabstract}

% \ekeywords{\TeX, \LaTeX, CJK, template, thesis}
