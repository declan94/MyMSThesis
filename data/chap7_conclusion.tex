\chapter{总结与展望}

\section{论文工作总结}

动物行为学是生物学研究中的重要领域,而果蝇行为学是动物行为学中的一个分支。果蝇因其形态简单等原因,可以通过计算机视觉、机器学习等相关领域的技术对果蝇进行自动识别。本文为了提高果蝇行为识别对视频拍摄环境变化的鲁棒性,针对果蝇行为识别的果蝇活动台分割和果蝇身体轮廓提取部分进行研究,并以网站的形式提供果蝇行为检测服务。下面是本文的主要工作和创新点。

首先,果蝇活动台的分割操作一般通过对形态学上检测来完成,匹配的检测果蝇活动台的形状来提取果蝇活动台的位置。为了适应不同果蝇行为视频的差异,本文通过对形态学操作中的参数控制形检测到的形状的数目,并对参数进行自适应,提高了果蝇活动台匹配算法的适用性。同时,针对特定排列的果蝇活动台,通过对活动台的实际坐标和理想坐标之间建立仿射变换,提高了果蝇活动台位置的准确性。

其次,针对果蝇身体轮廓提取算法,本文在背景模型的基础上,通过对果蝇视频背景进行首次建模,得到初步的果蝇身体和翅膀提取算法,进而通过分离果蝇身体和翅膀,实现对果蝇活动台的单独建模,得到精确的果蝇活动台模型。上述背景模型显著提高了果蝇身体轮廓提取的鲁棒性,在不同的果蝇行为视频中表现出相当的鲁棒性。

最后,通过搭建网站,提供果蝇行为识别的服务,可以方便不同研究人员使用相关的服务。

\section{未来工作展望}

基于本文开展的工作,将来的工作可以有以下几个考量:

\begin{enumerate}
\item 目前,本文仅针对果蝇活动台分割和果蝇身体轮廓提取部分进行了充分的实验,对于背景模型和果蝇行为识别的准确率之间的关系,受限于标定的果蝇视频的规模,目前还没有完全展开,希望以后能进一步在其他果蝇行为视频的数据上加以推广、验证。
\item 果蝇活动台分割算法的准确性还有待提高。对于排列不规则的果蝇活动台,对果蝇活动台中心进行聚类时偶尔会出现部分活动台被重复定位、部分活动台没有被准确定位到的问题,算法运行的结果还需要手工进行确认。
\item 果蝇行为分析的效率还存在提升的空间,目前在算法层面上可以改进的空间相对不多,可以用GPU运算等方式提高检测程序的处理能力。
\item 可以将对果蝇的背景模型等研究加以推广,应用到其他类型的研究中。
\end{enumerate}
