\thusetup{
  %******************************
  % 注意:
  %   1. 配置里面不要出现空行
  %   2. 不需要的配置信息可以删除
  %******************************
  %
  %=====
  % 秘级
  %=====
  secretlevel={绝密},
  secretyear={2100},
  %
  %=========
  % 中文信息
  %=========
  ctitle={短波语音信号的质量评价算法及自动选路系统研究},
  cdegree={工学硕士},
  cdepartment={电子工程系},
  cmajor={信息与通信工程},
  cauthor={陈晔},
  csupervisor={谷源涛副教授},
  %cassosupervisor={陈文光教授}, % 副指导老师
  %ccosupervisor={某某某教授}, % 联合指导老师
  % 日期自动使用当前时间,若需指定按如下方式修改:
  cdate={二〇一八年六月},
  %
  % 博士后专有部分
  %cfirstdiscipline={计算机科学与技术},
  %cseconddiscipline={系统结构},
  %postdoctordate={2009年7月——2011年7月},
  %id={编号}, % 可以留空: id={},
  %udc={UDC}, % 可以留空
  %catalognumber={分类号}, % 可以留空
  %
  %=========
  % 英文信息
  %=========
  etitle={Study on Quality Evaluation Algorithms and Automatic Switching Systems for Speech in HF Communications},
  % 这块比较复杂,需要分情况讨论:
  % 1. 学术型硕士
  %    edegree:必须为Master of Arts或Master of Science(注意大小写)
  %             “哲学、文学、历史学、法学、教育学、艺术学门类,公共管理学科
  %              填写Master of Arts,其它填写Master of Science”
  %    emajor:“获得一级学科授权的学科填写一级学科名称,其它填写二级学科名称”
  % 2. 专业型硕士
  %    edegree:“填写专业学位英文名称全称”
  %    emajor:“工程硕士填写工程领域,其它专业学位不填写此项”
  % 3. 学术型博士
  %    edegree:Doctor of Philosophy(注意大小写)
  %    emajor:“获得一级学科授权的学科填写一级学科名称,其它填写二级学科名称”
  % 4. 专业型博士
  %    edegree:“填写专业学位英文名称全称”
  %    emajor:不填写此项
  edegree={Master of Science},
  emajor={Information and Communication Engineering},
  eauthor={Chen Ye},
  esupervisor={Associate Professor Gu Yuantao},
  %eassosupervisor={Chen Wenguang},
  % 日期自动生成,若需指定按如下方式修改:
  edate={April, 2018},
  %
  % 关键词用“英文逗号”分割
  ckeywords={短波, 语音质量评价, 自动选路},
  ekeywords={Short-wave, Speech Quality Evaluation, Automatic Switching}
}


% 定义中英文摘要和关键字
\begin{cabstract}

  短波通信又称高频通信,是指使用频率范围在高频(HF)的无线电进行通信的方式,最早出现于1900年代。短波通信主要利用天波电离层反射,所以无需中继站即可实现远距离通信,具有机动性强、设备成本低、对基础设施依赖小等优点,因此广泛应用于广播、军事和应急救援等领域。而同时,短波通信的缺陷也非常突出,因为受电离层变化和多径传播等因素的影响,短波通信的信道非常不稳定,导致其通信质量起伏较大,影响通信的稳定性。
  
  在军事应用中,为了提高短波语音通信的通信质量和可靠性,常采用多路接收的措施。例如在空地通信中,在地面不同地点建立多个短波接收站台,再将接收到的短波语音信号汇总到一起,由人工选择一路质量最优的信号接入给地面指挥中心。人工选择的效率不高,稳定性不强,影响通信的及时性和可靠性,亟待由可靠的算法代替。目前对于语音质量的评价算法大多针对信噪比较高的语音,对短波语音并不能很好的适用。为此,本文提出了两种应用于短波语音的客观质量评价算法和一套自动选路系统。主要工作可以概括为以下几个方面:

  首先,本文针对短波语音的特殊性,提出了一种基于语谱图噪音模型的语音质量客观评价算法,通过对短波语音语谱图上的噪音谱建模、分析、综合,给出短波语音的客观质量评分。另外提出了一种基于自编码器的语音质量客观评价算法,可以更广泛地应用于其他特殊领域的语音质量评价。

  其次,本文提出了一种基于互相关和最小生成树的多路语音时域对齐算法,并在此基础上提出了多路短波语音自动选路切换系统。这是针对上述场景中人工选路系统的优化方案。

  再次,搭建了一套在线的语音质量主观评价辅助系统,并且利用该系统构建了一组具有平均主观意见分标签的短波语音数据集。

  最后,在短波语音数据上,本文将所提的客观质量评分算法和两种标准算法进行了效果比较,多项指标均反映出本文所提算法有着更好的表现。本文还进行了实际的多路语音选路的实验,验证了本文所提自动选路系统的功能。

  本文所提的算法和系统有着较高的实用价值,可以在实际应用中提高短波语音通信的质量和可靠性。本文所提的语音客观质量评价算法不仅适用于短波语音领域,也适用于对其他低信噪比语音的处理。

\end{cabstract}

% 如果习惯关键字跟在摘要文字后面,可以用直接命令来设置,如下:
% \ckeywords{\TeX, \LaTeX, CJK, 模板, 论文}

\begin{eabstract}

High-frequency (HF) communication, also known as short-wave communication, refers to the use of high frequency range in radio communication, which first appeared in the 1900s. HF communication exploits the ionospheric reflection to achieve long-distance communication without relay stations. It is widely used in radio broadcast, military, emergency relief and many other applications due to its flexibility, mobility, low cost of equipment, and low dependence on infrastructure. Meanwhile, a severe shortage of HF communication is that its communication channel is very unstable due to ionospheric changes and multipath propagation, which causes the communication quality to fluctuate, and reduces the stability of communication.

In military applications, in order to improve the quality and reliability of speeches in HF communication, multi-path reception measures are often applied. For example, in the case of communication between aircraft and ground control center, multiple HF receiving stations are established at different locations, receiving signals from aircraft. Then the one with the best quality among those multiple signals is chosed mannually and is transimitted to the ground control center. The mannual selection is not efficient and stable, which reduces timeliness and reliability of communication. It needs to be replaced by algorithms. At present, most of the speech quality evaluation algorithms are based on speech signals with high SNR, and are not suitable for HF speech. For this reason, this paper proposes two kinds of objective quality evaluation algorithms applied to HF speech and an automatic switching system. The main work can be summarized as the following aspects:

Firstly, aiming at the particularity of HF speech, this paper proposes an objective speech quality evaluation algorithm based on the 
noise spectrum model, scoring the quality of speech through the modeling, analysis and synthesis of the noise spectrum. In addition, another objective evaluation algorithm for speech quality based on autoencoder is proposed, which can be easily applied to speech quality evaluation in many other fields.

Secondly, this paper proposes a time alignment algorithm for multi-channel speeches based on cross-correlation and minimum spanning tree. Based on this, an automatic switching system for multi-channel HF speeches is proposed. This is an optimization scheme for the manual switching system in the above scenario.

Thirdly, an online assistant system for subjective quality evaluation of speeches is established, and a dataset of HF speech pieces with MOS (mean opinion score) is constructed with the help of the system.

Finally, this paper compares performance of the proposed algorithms with two standard speech quality evaluation algorithms on the HF speech dataset, which reflects the better performance of the proposed algorithms. This paper also carried out an actual multi-channel speech switching experiment to verify the function of the proposed system.

The algoritms and system proposed in this paper have high practical value and can improve the quality and reliability of HF speech communication in practical applications. The proposed objective quality evaluation algorithm for speech is not only applicable to HF speech, but also applicable to other low-SNR speech.

\end{eabstract}

% \ekeywords{\TeX, \LaTeX, CJK, template, thesis}
