\chapter{总结与展望}

\section{论文工作总结}

短波通信由于其具有机动性强、设备成本低、对基础设施依赖小的优点,被广泛应用于广播、军事和抢险救灾等领域。但是其信道不稳定,通信质量难以保证。在军事低空通信应用中,会在地面设立多个接收站,从不同地点接收来自空中飞机的短波语音信号,然后汇总到一起由人来选择一路质量最优的信号接入。从而提高短波语音通信的稳定性以及通信质量。

本文针对这一实际应用场景,试图使用算法替代人工选择的步骤。主要工作内容及创新点包括:

\begin{enumerate}
\item 提出了一种基于语谱图噪音模型的短波语音客观评价算法,通过在频域分析短波语音的语谱图给出短波语音的客观质量评分。创造性的提出使用一系列的阈值将连续的语谱图转化成一系列二值化的语谱图,转化成二值化是为了利于视觉算法进行处理,而采用一系列的不同阈值则是为了保留尽量多的信息。同时这也符合人耳从背景噪音中收集语音信号的模型,因为人在从带噪信号中分辨语音的信息时,正是主观地抑制某一能量层级以下的噪音能量,从而分辨出能量较高的语音部分。我们进一步基于此提出了“语音分辨率”的概念,通过计算语谱图上各点的“语音分辨率”给出语音的客观质量评分。
\item 提出了一种基于自编码器的短波语音客观评价算法,使用自编码器学习和记忆正常短波语音的语谱图结构,进而分析待评价短波语音的语谱畸变大小给出评分。将短波语音转换成梅尔频谱语谱图,再切分拉伸成固定长度的输入向量,使用Tensorflow实现了自编码器的训练和使用。然后将待评价的短波语音语谱图输入自编码器,通过比较输入语谱图和输出语谱图的差异来评判语音的质量高低。该算法旨在提供更加广泛适用的语音质量客观评价算法,以弥补算法1依赖于对语谱图上的噪音特征建模的缺陷。
\item 基于短波语音的客观评价算法进一步实现了短波语音自动选路系统,用以替代实际应用场景中的人工选路步骤。该系统自动均衡各路短波语音的功率,分析延时然后对齐各路语音,进行评分然后平滑地进行语音切换。
\item 建立了一套用于辅助短波语音主观质量评价的在线辅助系统,该系统可以通过与志愿者的互动,对一系列语音进行质量排序,或者给出语音的平均主观意见分。其中为了实现语音质量按照主观感受进行排序,提出了一种交互式实现的模糊快排算法。
\item 利用短波收音机采集了一系列短波语音的数据,建立了一套短波语音的数据集。包括一个按照主观质量排序的包含207个语音片段的数据集,用于短波语音客观质量评价等相关工作的研究;一个带有平均主观意见分标签的包含500个语音片段的数据集,用于短波语音质量客观评价算法的效果验证;一个多地点接收的包含20组,每组5至6路语音的多路短波语音数据集,用于自动选路系统的效果验证。
\item 通过实验验证了本文所提算法及系统的效果,将本文所提的短波语音客观质量评价算法与两种标准算法做了结果比较,表明本文所提算法有更好的效果。在实际的多路接收的短波语音上应用了本文所提的自动选路系统,表明系统能够正常实现选路切换并提升接收语音的质量。
\end{enumerate}

\section{未来工作展望}

基于本文开展的工作,将来的工作可以有以下几个考量:

\begin{enumerate}
\item A
\item B
\end{enumerate}
