\chapter{果蝇行为识别实验}\label{chap:fly_behavior_detection}

果蝇行为一般包括打架行为、求偶行为等,不同的行为有不同的特点。本章通过实验,在第\ref{chapter:fly_contour}章中果蝇轮廓提取算法的基础上,使用车向前提出的果蝇特征提取算法和果蝇行为识别算法\cite{chexiangqian},完成整个果蝇打架行为的检测。实验结果表明,计算机可以比人工更好的处理果蝇行为分析视频。

%%%%%%%%%%%%%%%%%%%%%%%%%%%%%%%%%%%%%%%%%%%%%%%%%%%%
%                   果蝇行为视频的数据标定
%%%%%%%%%%%%%%%%%%%%%%%%%%%%%%%%%%%%%%%%%%%%%%%%%%%%
\section{果蝇行为视频的数据标定} \label{sec:fly_behavior_character}

一般来说,果蝇行为学中关注的果蝇行为主要包括果蝇的打架行为、求偶行为等\cite{copulation_2009}。果蝇的打架行为和果蝇的求偶行为存在不同的外在特点,在进行果蝇行为识别中一般也采用不同的果蝇特征。本实验主要针对果蝇的打架行为进行检测。

实验所用的果蝇行为视频主要来自北京大学饶毅实验室,果蝇行为经过相关专家标定,具有较高的可信性和一致性。

%%%%%%%%%%%%%%%%%%%%%%%%%%%%%%%%%%%%%%%%%%%%%%%%%%%%
%                   果蝇行为视频的数据标定
%%%%%%%%%%%%%%%%%%%%%%%%%%%%%%%%%%%%%%%%%%%%%%%%%%%%
\section{果蝇行为识别的流程}

果蝇行为识别的步骤一般分为:果蝇活动台提取和分割、果蝇轮廓提取和追踪、果蝇特征提取、果蝇行为识别几个步骤。如图~\ref{fig:fly_detection_procedure}所示。

\begin{figure}
\centering
\begin{tikzpicture}
\node [block, text width=10em] (split_chambers) {\heiti 果蝇活动台分割};
\node [block, text width=10em, below = 0.5cm of split_chambers] (extract_contour) {\heiti 果蝇身体轮廓提取};
\node [block, text width=10em, below = 0.5cm of extract_contour] (feature_extract) {\heiti 果蝇特征提取};
\node [block, text width=10em, below = 0.5cm of feature_extract] (behavior_detect) {\heiti 果蝇行为分析};

\path [arrow] (split_chambers) -- (extract_contour);
\path [arrow] (extract_contour) -- (feature_extract);
\path [arrow] (feature_extract) -- (behavior_detect);
\end{tikzpicture}
\caption{果蝇行为识别自动化分析流程}
\label{fig:fly_detection_procedure}
\end{figure}

其中,第\ref{chapter:container_split}章介绍了果蝇活动台分割算法,介绍了如何从包含多个果蝇活动台的视频中切分出包含单个果蝇活动台的视频,在果蝇活动台排列存在一定规律时,还可以利用基于特定排列的果蝇活动台分割算法进行活动台分割。

根据第\ref{chapter:fly_contour}章介绍的果蝇轮廓提取算法,提取果蝇的身体轮廓。对某一个视频进行背景建模后,得到的模型对视频中的每一帧进行处理,得到视频帧中的果蝇身体和翅膀轮廓。求果蝇身体和翅膀区域的连通域,将其中面积太小的判定为噪声并删除,然后将连通区域划分成分离的果蝇。对分离的果蝇进行编号,并跟踪果蝇的运动情况,判定果蝇的头部和尾部。在此基础上,计算果蝇的特征。

果蝇的特征提取分为3步\cite{chexiangqian}:
\begin{enumerate}
\item 果蝇特征点的初步定位;
\item 果蝇特征点的SDM算法迭代更新;
\item 根据果蝇特征点进行果蝇姿态的重构。
\end{enumerate}

计算果蝇的特征点需要对果蝇进行椭球形建模,提取果蝇身体周围的8个特征点:果蝇的头部$\vec{p}_{h}$、尾部$\vec{p}_{t}$、左侧$\vec{p}_{l}$、右侧$\vec{p}_{r}$、背部$\vec{p}_{top}$、腹部$\vec{p}_{bottom}$、左翅膀根部$\vec{p}_{l,wing}$、右翅膀根部$\vec{p}_{r,wing}$。上述8个特征点中,果蝇的头部、尾部、左侧、右侧、背部、腹部主要根据果蝇身体连通域的外包矩形计算得到,果蝇的左翅膀根部和右翅膀根部通过翅膀轮廓处理得到,一般来说,翅膀表现出较尖、细的特征,在视频中不容易直接找到,可以分按照以下步骤进行:首先,找到果蝇翅膀和果蝇身体的连接处$\vec{p}_{lwing,root}$和$\vec{p}_{rwing,root}$;其次,计算翅膀的张开角度,对果蝇翅膀上的每一个像素点计算其相对翅膀根部的角度,并进行直方图统计,直方图中取值最大的角度就对应了果蝇翅膀的角度;最后,根据果蝇翅膀的长度和翅膀张开的角度,计算果蝇翅膀根据的位置$\vec{p}_{l,wing}$和$\vec{p}_{r,wing}$。

计算完成果蝇特征点后,采用SDM算法进行特征点的迭代更新,得到更准确的果蝇特征点。SDM算法最开始被用于人脸识别中人脸特征点的定位,车向前将SDM算法应用于果蝇特征点的迭代更新文章\cite{chexiangqian},取得了较好的效果。本文也采用该论文介绍的子空间SDM算法,对提取到的果蝇特征点位置进行更新。最后,使用子空间的SDM算法迭代更新后的果蝇特征点对果蝇的身体姿态进行重建,得到更精确的果蝇姿态,如果蝇翅膀倾角等。

最后,根据果蝇的姿态等特征,计算果蝇打架行为的特征。果蝇的打架行为包括直视、后挫、前冲等3个步骤,既包含了帧内果蝇之间的空间关系,还包含了时间上复杂的逻辑关系,因而果蝇打架行为的特征也包括了帧内特征和帧间特征\cite{chexiangqian}。对打架行为特征,经过数据标定,共得到10个视频、400个果蝇打架行为样本,对上述样本上训练SVM分类器,得到针对果蝇打架行为的分类器。

需要注意的是,上述果蝇特征点提取、利用SDM算法对果蝇特征点进行迭代更新、果蝇姿态重构、果蝇打架特征的计算等算法参照车向前论文\cite{chexiangqian}。

\section{果蝇行为识别结果}

经过训练和测试,在果蝇打架数据集上,使用5-fold交叉校验,果蝇打架行为分类器取得了的$84.3\%$准确率和$76.5\%$的召回率。

\section{小结}

经过试验,我们得到了完整的果蝇行为识别系统,该系统可以得到优于人工的标定结果。
